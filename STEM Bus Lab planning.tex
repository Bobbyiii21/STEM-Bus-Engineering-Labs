\documentclass{article}
\usepackage[colorlinks]{hyperref}
\usepackage{helvet}
\renewcommand{\familydefault}{\sfdefault}
\title{Ideas for STEM Bus Labs}
\author{Bobby R. Stephens III}
\date{26 July 2023}
\begin{document}
\maketitle
\section*{Lab 1: Demonstration to Machine Learning and Object Detection using JeVois and TensorFlow}
    \subsection*{Objectives}
        \begin{itemize}
            \item Demonstrate the use of the JeVois camera and its capabilities
            \item Allow students to see how machine learning can be used to detect objects
            \item Show the structure of a machine learning model and compare it to tools such as FaceID
        \end{itemize}
    \subsection*{Materials}
        \begin{itemize}
            \item JeVois camera, found at \href{https://www.jevoisinc.com/products/jevois-a33-smart-machine-vision-camera?variant=36249051658}{https://www.jevoisinc.com}
            \item Computer with JeVois software installed
            \item Objects to detect
        \end{itemize}
    \subsection*{Procedure}
        \begin{itemize}
            \item Show the students the JeVois camera and its software
            \item Explain how TensorFlow works and how it is used to detect objects
            \item Allow for students to experiment with the camera by using personal and provided objects
        \end{itemize}
\pagebreak
\section*{Lab 2: Introduction to Basic Robotics and \\Programming with VEX Go}
    \subsection*{Objectives}
        \begin{itemize}
            \item Introduce students robot systems found in the real world \\(Drivetrain, Arm)
            \item Convey the idea that robots and automation can be used to solve real-world problems
        \end{itemize}
    \subsection*{Materials}
        \begin{itemize}
            \item Pre-Built VEX Go Robot
            \item VEX Go Field 
            \item Simulated ``Trash'' (Manipulated trash bags with a rod through the middle, as well as another stretched across the top, to act as a handle)
            \begin{itemize}
                \item The bases will be 3D printed, and the existing magnets from the previous attempt will be used as weights
                \item Important measurements to note:
                \begin{itemize}
                    \item The handle will be at least 3.25 inches in diameter, to allow for the robot to pick it up
                \end{itemize}
            \end{itemize} 
        \end{itemize}
    \subsection*{Procedure}
        \begin{itemize}
            \item Show students the VEX Go system
            \item Demonstrate the robot's capabilities and how it can be programmed
            \item Task the students with either driving or programming the robot to move ``trash'' to designated areas on the VEX Go field
        \end{itemize} 
\pagebreak  
\section*{Lab 3: VEX v5 Transmission Demonstration}
    \subsection*{Objectives}
        \begin{itemize}
            \item To enhance the students' understanding of one of the fundemental \\systems of an automobile
            \item 
        \end{itemize}
    \subsection*{Materials}
        \begin{itemize}
            \item Coming Soon!
        \end{itemize}
    \subsection*{Procedure}
        \begin{itemize}
            \item Coming Soon!
        \end{itemize}
\pagebreak  
\setcounter{section}{4} 
\section*{Lab 4: Laws of Motion Labs- Simple Activities to Demonstrate a Greater Concept: Newton's First Law} 
    \subsection{The Law in practice}
        \begin{itemize} 
            \item Newton's first law states that an object in rest will stay in rest, and an object in motion will stay in motion unless acted upon by an external force
        \end{itemize}
    \subsection{Objectives}
        \begin{itemize}
            \item Coming Soon!
        \end{itemize}
    \subsection{Materials}
        \begin{itemize}
            \item Coming Soon!
        \end{itemize}
    \subsection{Procedure}
        \begin{itemize}
            \item Coming Soon!
        \end{itemize}
\pagebreak  
\setcounter{section}{5}
\setcounter{subsection}{0} 
\section*{Lab 5: Laws of Motion Labs- Simple Activities to Demonstrate a Greater Concept: Newton's Second Law}
    \subsection{The Law in practice}
        \begin{itemize} 
            \item Newton's second law states that the force of an object is directly \\proportional to an object's mass and acceleration
        \end{itemize}
    \subsection{Objectives}
        \begin{itemize}
            \item Coming Soon!
        \end{itemize}
    \subsection{Materials}
        \begin{itemize}
            \item Baloon filled with helium
            \item Electronic scale
            \item 
        \end{itemize}
    \subsection{Procedure}
        \begin{itemize}
            \item Coming Soon!
        \end{itemize}
\pagebreak
\setcounter{section}{6}
\setcounter{subsection}{0}
\section*{Lab 6: Laws of Motion Labs- Simple Activities to Demonstrate a Greater Concept: Newton's Third Law} 
    \subsection{The Law in practice}
        \begin{itemize} 
            \item Newton's third law states all forces come in pairs, and when one object pushes or pulls another object, the second object pushes or pulls back with an equal force in the opposite direction
        \end{itemize}
    \subsection{Objectives}
        \begin{itemize}
            \item Students will have a rudimentary understanding of Newton's third law of motion
        \end{itemize}
    \subsection{Materials}
        \begin{itemize}
            \item Cardboard Rectangle
            \item Drinking straws
            \item Skewers
            \item Balloons
        \end{itemize}
    \subsection{Procedure: Pre-Rollout}
        \begin{enumerate}
            \item Cut a rectangle out of cardboard
            \item Create a hole between the layers of the cardboard 1/4 of the lenth down the rectangle using the skewers
            \item Once a hole is created, insert the straw into the hole and cut it to size
            \item Repeat steps 2 and 3 approximately 3/4 down the length of the rectangle
            \item Complete the rest of the steps with students at the rollout\\
            \href{https://youtu.be/-OENIttg1dU}{Reference Video}
        \end{enumerate}
       
    \subsection{Procedure: Rollout}
        \begin{itemize}
            \item Retrieve the rectangle from pre-rollout
            \item Guide the students through the process of creating the axles of the car using skewers and bottle caps
            \item Add the balloon and straw to the top of the car
            \item Have fun!
        \end{itemize}
\end{document}
